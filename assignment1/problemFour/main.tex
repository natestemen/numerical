\documentclass[boxes,pages]{homework}

\name{Nate Stemen}
\studentid{20906566}
\email{nate.stemen@uwaterloo.ca}
\term{Fall 2020}
\course{Numerical Analysis}
\courseid{AMATH 740}
\hwnum{2}
\duedate{Fri, Sep 25, 2020 5:00 PM}

\hwname{Assignment}
% \problemname{Exercise}
%\solutionname{(Name)}


\usepackage{physics}
\usepackage{cleveref}

\newcommand{\tpose}[1]{#1^\intercal}
\newcommand{\inv}[1]{#1^{-1}}


\begin{document}

\problemnumber{4}

\begin{problem}
Conditioning of linear systems.
\end{problem}

\begin{solution}
	\subsubsection*{(a)}
	We begin by showing $\kappa_2(A)$ is always equal to 1 if $A$ is an orthogonal matrix.
	\begin{equation}\label{eq:condition}
		\kappa_2(A) = \norm{A}_2\norm*{\inv{A}}_2
	\end{equation}
	We can expand these norms out as follows where we use $\inv{A} = \tpose{A}$ because for an othognal matrix $A\tpose{A} = \tpose{A}A = \mathbf{1}$.
	\begin{align*}
		\norm{A}_2^2 & = \lambda_\text{max}\qty(\tpose{A}A) & \norm*{\inv{A}}_2^2 & = \lambda_\text{max}\qty(\tpose{\qty(\inv{A})}\inv{A}) \\
		             & = \lambda_\text{max}\qty(\mathbf{1}) &                     & = \lambda_\text{max}\qty(A\tpose{A})                   \\
		             & = 1                                  &                     & = \lambda_\text{max}\qty(\mathbf{1}) = 1
	\end{align*}
	Plugging these two equations into \cref{eq:condition} we see $\kappa_2(A) = 1$ for all orthogonal $A$.

	\subsubsection*{(b)}
	Here we show the matrix $A = \smqty[1 & -1 \\ 1 & \phantom{-}1]$ is not orthogonal.
	\begin{equation*}
		A\tpose{A} = \mqty[1 & -1 \\ 1 & \phantom{-}1]\mqty[\phantom{-}1 & 1 \\ -1 & 1] = \mqty[2 & 0 \\ 0 & 2] = 2\mathbf{1} \neq \mathbf{1}
	\end{equation*}
	We can see here that $\inv{A} = \frac{1}{2}\tpose{A}$ which we will use below in our calculation of $\kappa_2(A)$.
	\begin{align*}
		\kappa_2(A) & = \norm{A}_2\norm*{\inv{A}}_2 = \sqrt{\lambda_\text{max}\qty(\tpose{A}A)}\sqrt{\lambda_\text{max}\qty(\tpose{\qty(\inv{A})}\inv{A})} \\
		            & = \sqrt{\lambda_\text{max}\qty(2\cdot\mathbf{1})}\sqrt{\tfrac{1}{4}\lambda_\text{max}\qty(A\tpose{A})}                               \\
		            & = \sqrt{2}\sqrt{\tfrac{1}{2}} = 1
	\end{align*}
	Because $\kappa_2(A) = 1$, we can conclude the matrix is well-conditioned.

	When computing the solution to the perturbed problem, the change in solution from the unperturbed problem is very small. This is because the matrix is well conditioned ($\kappa_2$ is small)

	\subsubsection*{(c)}
	First, is the matrix $B = \smqty[1 & -1 + \delta \\ 1 & -1]$ orthogonal?
	\begin{equation*}
		B\tpose{B} = \mqty[1 & -1 + \delta \\ 1 & -1]\mqty[1 & 1 \\ -1 + \delta & -1] = \mqty[1 + \qty(\delta - 1)^2 & 2 - \delta \\ 2 - \delta & 2] \neq \mathbf{1}
	\end{equation*}
	We conclude the matrix is not orthogonal. Via python we can calculate the matrix condition number to be $\kappa_2(B) = 39,999,947,698.45045$, and hence we conclude the matrix is not well conditioned.

	When computing the solution to the perturbed problem, the change in solution from the unperturbed problem is massive (on the order $10^6$). This is because the matrix is very ill conditioned ($\kappa_2$ is \emph{very} large).

\end{solution}

\end{document}